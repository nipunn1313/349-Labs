\documentclass{article}

\usepackage[utf8]{inputenc}
\usepackage{listings}
\usepackage[margin=1in]{geometry}
\usepackage{graphicx, hyperref, mathpazo, units, breakurl}

\setlength{\parindent}{0in}
\hypersetup{colorlinks=true}

\newcommand{\asm}[1]{\texttt{#1}}
\long\def\omitit#1{}

\title{\huge Lab 1: Introduction to Gumstix and Code Optimization 
       - Verdex Setup}
\author{\Large\itshape 18--349 Embedded Real-Time Systems}

\begin{document}
	\maketitle

	\section{Creating a bootable microSD card.}

   What you will need?  \\

   \begin{enumerate}
   \item A development machine with some version of linux (I had Ubuntu 9.04
         installed on my machine) .  You should have root access on  this 
         machine (or at least the ability to \texttt{mount/unmount} devices and
         run \texttt{fdisk}).  
   \item A  microSD card adapter. Your hardware should contain an adapter that
         converts microSD card into an SD card which you can use on your
         computer. If your computer/laptop does not have an SD card reader,
         please contact the course staff  to get a USB adapter for the microSD
         card.   
   \end{enumerate}

   The example covered in this document will show the steps for setting up 
   a brand new 2GB microSD card. \\

   First insert your card into your development machine's flash card slot. \\

   On my Ubuntu 9.04 machine, the newly inserted card shows up as
   \texttt{/dev/sdb1} and that  is the device name that will be used through
   this example. You should substitute the proper device name for your machine.
   You can use '\texttt{mount}' or '\texttt{df}' to see where the card mounts
   on your machine. 

   \subsection{Step 1}
   Unmount the device's existing file system before we get started
   with \texttt{fdisk} 
	\begin{verbatim} 
	% svn add file1 file2 file3 
	\end{verbatim}

   \section{Partitioning the card}
   \subsection{Step 2}
      Now launch \texttt{fdisk} and create an empty partition table. Note that
      the argument for \texttt{fdisk} is the entire device \texttt{(/dev/sdb)}
      not just a single partition \texttt{(i.e. /dev/sdb1):} 
      \begin{verbatim}
      # sudo fdisk /dev/sdb 
      Command (m for help): o 
      Building a new DOS disk label. Changes will remain in memory only, until 
      you decide to write them. After that, of course, the previous content 
      won't be recoverable. 
      Warning: invalid flag 0x0000 of partition table 4 will be corrected by
      w(rite) 
      \end{verbatim}
   \subsection{Step 3}
      First look at the current card information: 
      \begin{verbatim}
      Command (m for help): p 
      Disk /dev/sde: 2032 MB, 2032664576 bytes 
      64 heads, 63 sectors/track, 984 cylinders 
      Units = cylinders of 4032 * 512 = 2064384 bytes 
      Disk identifier: 0x00aa8e5c 
         Device Boot      Start         End      Blocks   Id  System
      \end{verbatim}
      Note the card size in bytes. We will needed it later in the process.
   \subsection{Step 4}
      Now go into "Expert" mode: 
      \begin{verbatim}
      Command (m for help): x
      \end{verbatim}
   \subsection{Step 5}
      Next we will set the geometry to 255 heads, 63 sectors and a calculated
      value for the number of cylinders required for the particular microSD
      card. \\
      
      To calculate the number of cylinders, we take the 2032664576 (replace
      this number by the number reported by your \texttt{fdisk} command)
      bytes reported above by \texttt{fdisk} divided by 255 heads, 63 sectors
      and 512 bytes per sector: \\

      2032664576 / 255 / 63 / 512 = 247.12 which we round down to 247
      cylinders.
   \subsection{Step 6}
     Set the number of heads, sectors and cylinders
     \begin{verbatim}
     Expert command (m for help): h 
     Number of heads (1-256, default 4): 255 

     Expert command (m for help): s 
     Number of sectors (1-63, default 62): 63 
     Warning: setting sector offset for DOS compatiblity 

     Expert command (m for help): c 
     Number of cylinders (1-1048576, default 984): 247
    \end{verbatim}
   \subsection{Step 7}
      Return to \texttt{fdisk's} main mode and create a new partition 
      \texttt{32 MB FAT} partition: 
      \begin{verbatim}
      Expert command (m for help): r 
      Command (m for help): n 
      Command action 
         e   extended 
         p   primary partition (1-4) 
      p 
      Partition number (1-4): 1 
      First cylinder (1-247, default 1): 1 
      Last cylinder or +size or +sizeM or +sizeK (1-247, default 15): +32M
      \end{verbatim}
   \subsection{Step 8}
      Change the partition type to \texttt{FAT32}: 
      \begin{verbatim}
      Command (m for help): t 
      Selected partition 1 
      Hex code (type L to list codes): c 
      Changed system type of partition 1 to c (W95 FAT32 (LBA)) 
      \end{verbatim}

      And mark it bootable: 

      \begin{verbatim}
      Command (m for help): a 
      Partition number (1-4): 1
      \end{verbatim}
   \subsection{Step 9}
      Next we create an \texttt{ext3} partition for the rootfs. 
      \begin{verbatim}
      Command (m for help): n 
      Command action 
         e   extended 
         p   primary partition (1-4) 
      p 
      Partition number (1-4): 2 
      First cylinder (6-247, default 6): 6 
      Last cylinder or +size or +sizeM or +sizeK (6-247, default 247): 247
      \end{verbatim}
   \subsection{Step 10}
      To verify our work, lets print the partition info 
      \begin{verbatim}
      Command (m for help): p 

      Disk /dev/sdb: 2032 MB, 2032664576 bytes 
      255 heads, 63 sectors/track, 247 cylinders 
      Units = cylinders of 16065 * 512 = 8225280 bytes 
      Disk identifier: 0x00aa8e5c 

         Device Boot      Start         End      Blocks   Id  System 
      /dev/sdb1   *           1           5       40131    c  W95 FAT32 (LBA) 
      /dev/sdb2               6         247     1943865   83  Linux 
      \end{verbatim}
   \subsection{Step 11}
      Up to this point no changes have been made to the card itself, so our
      final step is to write the new partition table to the card and then exit.
      \begin{verbatim}
      Command (m for help): w 
      The partition table has been altered! 

      Calling ioctl() to re-read partition table. 

      WARNING: If you have created or modified any DOS 6.x partitions, please 
      see the fdisk manual page for additional information. 

      Syncing disks.
      \end{verbatim}
   \section{Formatting the new partitions}
   \subsection{Step 12}
      We format the first partition as a FAT file system (the -n parameter
      gives it a label of FAT, you can change or omit this if you like)
      \begin{verbatim}
      #  sudo mkfs.vfat -F 32 /dev/sdb1 -n FAT 
      mkfs.vfat 2.11 (12 Mar 2005) 
      \end{verbatim}
   \subsection{Step 13}
      We format the second partition as an \texttt{ext3} file system
      \begin{verbatim} 
      $ sudo mkfs.ext3 /dev/sdb2 
      mke2fs 1.40.8 (13-Mar-2008) 
      Filesystem label= 
      OS type: Linux Block size=4096 (log=2) 
      Fragment size=4096 (log=2) 
      121920 inodes, 485966 blocks 
      24298 blocks (5.00%) reserved for the super user 
      First data block=0 
      Maximum filesystem blocks=499122176 
      15 block groups 
      32768 blocks per group, 32768 fragments per group 
      8128 inodes per group 
      Superblock backups stored on blocks:  
      32768, 98304, 163840, 229376, 294912 

      Writing inode tables: done                             
      Creating journal (8192 blocks): done 
      Writing superblocks and filesystem accounting information: ^[done 

      This filesystem will be automatically checked every 36 mounts or 
      180 days, whichever comes first.  Use tune2fs -c or -i to override. 
      \end{verbatim}
   \section{Installing the boot files}
      There are 2 files required on the first (FAT) partition to boot your
      Verdex-Pro
      \begin{enumerate}
      \item gumstix-factory.script
      \item uimage: the linux kernel
      \end{enumerate}
   \subsection{Step 14}
      Mount the FAT partition of your microSD card. This example will assume
      that you have mounted it at \texttt{/media/card} 
      \begin{verbatim}
      $ sudo mount /dev/sdb1 /media/card 
      $ sudo cp gumstix-factory.script /media/card/gumstix-factory.script 
      $ sudo cp umage /media/card/uimage 
      $ sync; sync; sync; 
      \end{verbatim}
      The \texttt{sync} command will flush any of the buffered writes to the
      microSD card (the command may take a few seconds to execute). 
   \subsection{Step 15}
      You can now unmount the FAT partition: 
      \begin{verbatim}
      $ sudo umount /dev/sdb1 
      \end{verbatim}
      At this point you have a bootable FAT partition. 
   \subsection{Step 16}
      The final step is to untar your desired rootfs onto the \texttt{ext3}
      partition that you created above. \\

      Note that this step can be dangerous.  You do not want to untar your
      verdex-pro rootfs onto your development machine  by mistake. Be
      careful! \\

      This example will assume that you have mounted it at \texttt{/media/card}
      \begin{verbatim}
      $ sudo mount /dev/sdb2 /media/card 
      \end{verbatim}
      Now untar your desired rootfs: 
      \begin{verbatim}
      $ cd /media/card 
      $ sudo tar xvf /path/to/rootfs.tar.gz  
      $ sync; sync; sync; 
      \end{verbatim}

      The \texttt{sync} command will flush any of the buffered writes to the
      microSD card (the command may take a few seconds to execute).
   \subsection{Step 17}
      You can now unmount the \texttt{ext3} partition: 
      \begin{verbatim}
      $ sudo umount /dev/sdb2 
      \end{verbatim}

      Remove the card from the adapter and insert it in the verdex-pro’s card
      slot. Apply power to the hardware board. If you see a message saying
      that the system is booting from the microSD card, Congratulations!! If
      not, you probably made an error somewhere. Try once again and if not
      successful, contact your TA or me \url{rgandhi@ece.cmu.edu}. Once the system
      boots up of the microSD card, login using “\texttt{root}” as userid and 
      “\texttt{gumstix}” as password. Write a simple C program and see if gcc
      works.  

      References: 
      \begin{enumerate}
      \item Gumstix developer’s site.
            \url{http://www.gumstix.net/User-How-To-s/view/User-How-To-s/Bootingfrom-microSD/SD/MMC/110.html}
      \end{enumerate}

\end{document}
